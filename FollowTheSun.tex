\documentclass{article}
\usepackage{amsmath}
\usepackage{amssymb}
\usepackage{enumerate}
\usepackage[margin = .75in]{geometry}
\usepackage{graphicx}
\usepackage{caption}
\usepackage{float}

\begin{document}

\title{Follow the Sun: Learning North from Images}
\author{Brandon Minor | Jack Morrison}
\maketitle

%%%%%%%%%%%%%%%%%%
\section{Abstract}

Humans have navigated by the sun for millennia, relying on its predictable path across the sky to determine their own heading relative to north. We have applied Machine Learning algorithms to this practice in an attempt to do the same through static images. Data was collected from three different areas in the US, all with different latitudes. 
%% Fill in more later

%%%%%%%%%%%%%%%%%%
\section{Methodology}

\subsection{Data Collection}
To train our parameters, our datasets need to include the heading and timestamp of every image. We produced an Android application to facilitate this process. Images were captured as fast as possible by the phone's camera and saved on external memory, along with the magnetometer data that the phone registered at that timestamp. The magnetometer registered the difference, in degrees, of the heading of the phone with respect to gravity and magnetic north. Images were only captured if the phone was nearly vertical, i.e. pitch of the phone was within $\pm$2$^\circ$. The user would spin in place until enough photos were registered. 

\subsection{}





%%%%%%%%%%%%%%%%%%
\section{Learning}






%%%%%%%%%%%%%%%%%%
\section{Results}









\end{document}